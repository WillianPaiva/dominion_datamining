%%%%%%%%%%%%%%%%%%%%%%%%%%%%%%%%%%%%%%%%%
%
%%%%%%%%%%%%%%%%%%%%%%%%%%%%%%%%%%%%%%%%%

%----------------------------------------------------------------------------------------
%	PACKAGES AND OTHER DOCUMENT CONFIGURATIONS
%----------------------------------------------------------------------------------------

\documentclass{article}

\usepackage{fancyhdr} % Required for custom headers
\usepackage{lastpage} % Required to determine the last page for the footer
\usepackage{extramarks} % Required for headers and footers
\usepackage[usenames,dvipsnames]{color} % Required for custom colors
\usepackage{graphicx} % Required to insert images
\usepackage{listings} % Required for insertion of code
\usepackage{courier} % Required for the courier font
\usepackage{lipsum} % Used for inserting dummy 'Lorem ipsum' text into the template
\usepackage[utf8]{inputenc}
\usepackage[T1]{fontenc}
\usepackage{lmodern}
\usepackage{verbatim}



% Margins
\topmargin=-0.45in
\evensidemargin=0in
\oddsidemargin=0in
\textwidth=6.5in
\textheight=9.0in
\headsep=0.25in

\linespread{1.1} % Line spacing

% Set up the header and footer
\pagestyle{fancy}
\lhead{\hmwkAuthorName} % Top left header
\rhead{\firstxmark} % Top right header
\lfoot{\lastxmark} % Bottom left footer
\cfoot{} % Bottom center footer
\rfoot{Page\ \thepage\ of\ \protect\pageref{LastPage}} % Bottom right footer
\renewcommand\headrulewidth{0.4pt} % Size of the header rule
\renewcommand\footrulewidth{0.4pt} % Size of the footer rule

\setlength\parindent{0pt} % Removes all indentation from paragraphs


%----------------------------------------------------------------------------------------
%	DOCUMENT STRUCTURE COMMANDS
%	Skip this unless you know what you're doing
%----------------------------------------------------------------------------------------

% Header and footer for when a page split occurs within a problem environment
\newcommand{\enterProblemHeader}[1]{
\nobreak\extramarks{#1}{#1 continued on next page\ldots}\nobreak
\nobreak\extramarks{#1 (continued)}{#1 continued on next page\ldots}\nobreak
}

% Header and footer for when a page split occurs between problem environments
\newcommand{\exitProblemHeader}[1]{
\nobreak\extramarks{#1 (continued)}{#1 continued on next page\ldots}\nobreak
\nobreak\extramarks{#1}{}\nobreak
}

\setcounter{secnumdepth}{0} % Removes default section numbers
\newcounter{homeworkProblemCounter} % Creates a counter to keep track of the number of problems

\newcommand{\homeworkProblemName}{}
\newenvironment{homeworkProblem}[1][Problem \arabic{homeworkProblemCounter}]{ % Makes a new environment called homeworkProblem which takes 1 argument (custom name) but the default is "Problem #"
\stepcounter{homeworkProblemCounter} % Increase counter for number of problems
\renewcommand{\homeworkProblemName}{#1} % Assign \homeworkProblemName the name of the problem
\section{\homeworkProblemName} % Make a section in the document with the custom problem count
\enterProblemHeader{\homeworkProblemName} % Header and footer within the environment
}{
\exitProblemHeader{\homeworkProblemName} % Header and footer after the environment
}

\newcommand{\problemAnswer}[1]{ % Defines the problem answer command with the content as the only argument
\noindent\framebox[\columnwidth][c]{\begin{minipage}{0.98\columnwidth}#1\end{minipage}} % Makes the box around the problem answer and puts the content inside
}

\newcommand{\homeworkSectionName}{}
\newenvironment{homeworkSection}[1]{ % New environment for sections within homework problems, takes 1 argument - the name of the section
\renewcommand{\homeworkSectionName}{#1} % Assign \homeworkSectionName to the name of the section from the environment argument
\subsection{\homeworkSectionName} % Make a subsection with the custom name of the subsection
\enterProblemHeader{\homeworkProblemName\ [\homeworkSectionName]} % Header and footer within the environment
}{
\enterProblemHeader{\homeworkProblemName} % Header and footer after the environment
}

%----------------------------------------------------------------------------------------
%	NAME AND CLASS SECTION
%----------------------------------------------------------------------------------------

\newcommand{\hmwkTitle}{Dominion datamining: expression du besoin} % Assignment title
\newcommand{\hmwkDueDate}{Friday,\ January\ 29,\ 2016} % Due date
\newcommand{\hmwkClass}{PdP} % Course/class
\newcommand{\hmwkClassInstructor}{Narbel Philippe, Boussicault Adrien} % Teacher/lecturer
\newcommand{\hmwkAuthorName}{VER VALEM Willian , BAYOL Elmer , Liliana Lopez Farfan , TAGNAOUTI Khaoula} % Your name

%----------------------------------------------------------------------------------------
%	TITLE PAGE
%----------------------------------------------------------------------------------------

\title{
\vspace{2in}
\textmd{\textbf{\hmwkClass:\ \hmwkTitle}}\\
\normalsize\vspace{0.1in}\small{Due\ on\ \hmwkDueDate}\\
\vspace{0.1in}\large{\textit{\hmwkClassInstructor}}
\vspace{3in}
}

\author{\textbf{\hmwkAuthorName}}
\date{} % Insert date here if you want it to appear below your name

%----------------------------------------------------------------------------------------

\begin{document}

\maketitle

%----------------------------------------------------------------------------------------
%	TABLE OF CONTENTS
%----------------------------------------------------------------------------------------

%\setcounter{tocdepth}{1} % Uncomment this line if you don't want subsections listed in the ToC

\newpage
\tableofcontents
\newpage

%----------------------------------------------------------------------------------------
%	PROBLEM 1
%----------------------------------------------------------------------------------------

% To have just one problem per page, simply put a \clearpage after each problem

\begin{homeworkProblem}[Présentation du projet]
  %Ce projet, proposé par Yvan Le Borgne du Labri, a pour but d'analyser des logs du jeu de cartes Dominion. En effet pendant plusieurs années, un serveur a été mis à disposition afin de pouvoir jouer au jeu de cartes. Les objectifs du projet sont d'obtenir une représentation des données utilisables (les logs sont très verbeux, au format html et ne respectent pas toujours la même structure) puis de proposer des moyens de représentation de ces données afin d'en extraire des observations. Ces observations seront variées, mis à part les statistiques relatives au déroulement des parties, le projet vise également à discerner les différentes stratégies utilisées par les joueurs, pour cela un wiki est à disposition et les critères à reconnaître découlent de celui-ci.

  Le jeu de cartes \textit{Dominion} a été mis à disposition sur un serveur de jeu d'octobre 2010 à mars 2013. Les parties effectuées via ce serveur ont été enregistrées dans des \textit{logs} qui mémorisaient toutes les actions des joueurs ; ces \textit{logs} ont été mis à disposition du public. Mais ces enregistrements n'ont pas conservé un certain nombre d'informations ayant un rapport avec les décisions prises par les joueurs.
  \newline Un wiki relatif au jeu a été élaboré, et propose des avis d'experts comme aide à la décision pour les joueurs : il propose des conseils, notamment au niveau stratégique. Le but de ce projet, proposé par Yvan Le Borgne, chercheur au Labri, est de traiter ces \textit{logs} (soit plus de 12 millions de parties) afin de répondre à plusieurs interrogations. Il s'agira principalement de comparer Les données recueillies aux préconisations de ce wiki,afin de valider, à travers les contenus des parties, l'éventuelle efficacité des avis et suggestions fournies par les experts. A tout le moins, on cherchera à élaborer des outils de validation présentant une efficacité suffisante.
   Dans cette optique, le projet a pour but de créer une base de données recensant les parties puis un travail d'analyse sera effectuée à partir de celle-ci.
  Pour cela notre programme devra tout d'abord extraire les données présentes dans les logs, car ceux-ci sont trop volumineux et trop compliqués à utiliser directement (structure non standardisée). Il s'agira de mettre au point des structures de données permettant de stocker les données de manière plus efficace. Une manière simple de représenter les données extraites et analysées sera proposée à l'utilisateur.
  Par ailleurs, il manque des informations dans les enregistrements (\textit{logs}). On cherchera à trouver par quelle méthode on peut les reconstituer et de les mettre dans un format qui contienne toutes les informations.
  En outre, on cherchera à optimiser les opérations menées dans l'analyse des donnnées, en recherchant le meilleur équilibre possible entre la mémorisation des recherches et le re-calcul.
  Enfin, on esssaiera de déterminer quelle stratégie a été utilisée dans chaque partie. Voire de faire émerger les changements de stratégie en cours de partie. Si on prend pour exemple une des stratégies (la \textit{Penultimate Province Rule}) peut on détecter à quel moment cette règle a été appliquée ou contournée ? ou qui a mis au point cette stratégie ? (découverte qui pourrait permettre d'organiser un classement des joueurs les plus performants) (une sorte de ELO demandé par le client). Dans la mesure où on parviendrait à mettre au point un nombre suffisant de ces démarches stratégiques, il s'agirait de déterminer le processus habituel d'apprentissage des joueurs.


\end{homeworkProblem}

\begin{homeworkProblem}[Besoins non-fonctionnels]
  \begin{homeworkSection}{Performances}
    Pour ce qui concerne l'extraction des données à partir des \textit{logs}, la durée de l'opération n'est pas cruciale pour l'exécution du projet. Néanmoins, elle devra rester dans les limites du raisonnable (par exemple 1 jour) car la taille des \textit{logs} atteint les 400 Go.
    En ce qui concerne l'analyse des données extraites, un temps d'exécution plus bas est demandé. En l'absence du traitement d'un premier lot de données pouvant servir à une évaluation, il est difficile de déterminer un temps d'exécution, même approximatif.
  \end{homeworkSection}
  \begin{homeworkSection}{Fiabilité}
    La quantité de parties enregistrées étant importante, on peut admettre de rejeter de 5 à 10\% de \textit{logs} inexploitables.
    \newline L'objectif concernant l'extraction des logs est de ne perdre aucune information. En ce qui concerne la reconstitution des informations manquantes dans les \textit{logs}, l'équipe projet se réserve de réaliser cette partie en fonction du temps restant après la réalisation de la première phase du projet.
  \end{homeworkSection}
  \begin{homeworkSection}{Facilité d'utilisation}
    Le programme pourra être constitué d'une invite de commandes, une interface graphique n'est pas requise. Les statistiques extraites des données devront, quant à elles, être représentées graphiquement. Il faudra créer des tests destinés à s'assurer que les informations extraites des \textit{logs} correspondent à 100/100 aux données fournies.
  \end{homeworkSection}
  \begin{homeworkSection}{Domaine d'action}
    Les données traitées consistent en des \textit{logs} d'un jeu de carte, ces "logs" pouvant être transcrits de manière diverse. Les utilisateurs du programme seront intéressés par les statistiques pouvant être obtenues à partir des données.
  \end{homeworkSection}

  \begin{homeworkSection}{Portabilité}
    Le programme devra fonctionner sur un environnement Linux, aucun autre sytème d'exploitation n'est envisagé.
  \end{homeworkSection}

  \begin{homeworkSection}{Taille des données}
    Les logs étant très volumineux, le premier enjeu sera d'en extraire les données tout en réduisant drastiquement la taille de ceux-ci, une taille inférieure à 20Go est la contrainte à respecter.
    \end{homeworkSection}

\end{homeworkProblem}

\begin{homeworkProblem}[Besoins fonctionnels]
  \begin{homeworkSection}{Parsing}
    Le programme doit pouvoir lire des \textit{logs} de parties et récupérer les informations contenues dans ces logs.
    \begin {itemize}
    \item Le parser doit pouvoir s'adapter aux différences entre les parties (informations manquantes). Il faut donc que le programme détecte si certaines portions du log sont manquantes et si possible, les reconsituer.
    \item Le parser doit accepter tous les logs dans la mesure du possible, dans certains cas de figures le parser pourra ignorer certaines parties (noms invalides, ambigus)
    \item Le parser va travailler par tranches de données, une tranche correspond à une journée de logs.
    \item Le parser devra fournir le nombre de parties traitées par rapport au nombre de parties en entrée.
    \end {itemize}
  \end{homeworkSection}
  \begin{homeworkSection}{Stockage}
    Le stockage se fera a l'aide de deux bases de données, une première de type no-SQL, une seconde de type relationnelle. Le système de stockage contiendra des requètes persistantes afin de faciliter l'accès aux données utiles.
  \end{homeworkSection}
  \begin{homeworkSection}{Classement des joueurs}
    Le programme devra fournir un moyen d'obtenir le classement des joueurs. La solution retenue consiste en l'utilisation du système d'Elo. Pour chaque partie il faudra connaitre l'elo initial de chaque joueur et calculer leur elo à l'issue de la partie.
  \end{homeworkSection}
  \begin{homeworkSection}{Stratégies utilisées}
    Le programme devra reconnaître la stratégie suivie par chaque joueur (si le joueur en suit une) dans une partie. Les stratégies sont décrites dans la page wiki du jeu.
  \end{homeworkSection}
  \begin{homeworkSection}{Greening}
    Pour terminer une partie, les joueurs doivent acheter des points de victoire, le programme devra reconnaître à quel moment ces points commencent à être achetés
  \end{homeworkSection}
  \begin{homeworkSection}{Visualisation des données}
    Il faut pouvoir représenter graphiquement les statistiques observées.
    \begin {itemize}
    \item Des statistiques intéressantes pourront être proposées par défaut et représentées directement à la demande.
    \item L'utilisateur pourra également créer ses propres statistiques et les représenter à l'aide d'un programme dédié à cette tache.
    \end {itemize}
  \end{homeworkSection}
\end{homeworkProblem}

\end{document}
