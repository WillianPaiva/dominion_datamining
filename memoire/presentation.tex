\chapter{Présentation du projet}

Le jeu de cartes \textit{Dominion} a été mis à disposition, sur un serveur de jeu, d'octobre 2010 à mars 2013. Les parties effectuées via ce serveur ont été enregistrées dans des \textit{logs} qui mémorisaient toutes les actions des joueurs ; ces \textit{logs} ont été mis à disposition du public. Mais ces enregistrements n'ont pas conservé un certain nombre d'informations ayant trait à des décisions prises par les joueurs.\\

Un wiki relatif au jeu a été élaboré ; il offre des avis d'experts susceptibles de servir d'aide à la décision pour les joueurs. Il propose des conseils, notamment au niveau stratégique. Le but de ce projet, proposé par Yvan Le Borgne, chercheur au Labri, est de traiter ces \textit{logs} (soit plus de 12 millions de parties) afin de répondre à plusieurs interrogations. Il s'agira principalement de comparer Les données recueillies aux préconisations de ce wiki, afin de valider, à travers les contenus des parties, l'éventuelle efficacité des avis et suggestions fournies par les experts. A tout le moins, on cherchera à élaborer des outils de validation présentant une efficacité suffisante.\\
 
A cet effet, le projet a pour but de créer une base de données recensant les parties ; puis un travail d'analyse sera effectué à partir de cette base de données.
Pour cela, notre programme devra tout d'abord extraire les données présentes dans les logs, car ceux-ci sont trop volumineux et trop compliqués pour pouvoir les utiliser directement (structure non standardisée). Il s'agira donc, dans un premier temps, de mettre au point des structures de données permettant de les stocker de façon plus efficace. Une manière simple de représenter les données extraites et analysées sera proposée à l'utilisateur.\\

Par ailleurs, il manque des informations dans les enregistrements (\textit{logs}). On cherchera à trouver par quelle méthode on peut les reconstituer et les mettre dans un format contenant toutes les informations.
En outre, on cherchera à optimiser les opérations menées dans l'analyse des donnnées, en recherchant le meilleur équilibre possible entre la mémorisation des recherches, et le re-calcul.\\

Enfin, on esssaiera de déterminer quelle stratégie a été utilisée dans chaque partie, voire de faire émerger les changements de stratégie en cours de partie. Si on prend par exemple une des stratégies (la \textit{Penultimate Province Rule}, peut-on détecter à quel moment cette règle a été appliquée ou contournée ? ou qui a mis au point cette stratégie ? (découverte qui pourrait permettre d'organiser un classement des joueurs les plus performants) (une sorte de ELO\footnotemark demandé par le client). Dans la mesure où on parviendrait à mettre au point un nombre suffisant de ces démarches stratégiques, il s'agirait de déterminer le processus habituel d'apprentissage des joueurs.

\footnotetext{le système de prédiction ELO, mis au point par le Pr Arpad Elo est un système mathématique simple qui permet de prédire la probabilité qu'un joueur d'échec, ou une équipe de baseball ou de basket, en gagne un(e) autre}

\section{Règles du jeu}
Chaque joueur possède un deck de cartes et a accès à un <<marché>> où différentes cartes d'action sont disponibles. Il y a également une pile de "cartes bonus" (en anglais \textit{victory cards}.\\

\section{Introduction au datamining}
%TODO: parler du datamining


\section{Sujet}

Bla(cf. fig. 1.1)\\

%inclusion d'une mage dans le document
\begin{figure}[!h]
\begin{center}
%taille de l'image en largeur
%remplacer "width" par "height" pour régler la hauteur
\includegraphics[width=15cm]{presentation/schema}
\end{center}
%légende de l'image
\caption{Schéma descriptif}
\end{figure}

%Contenu de la note précédemment marquée avec \footnotemark
\footnotetext{Note bas de page "intro"}

Bla
%retour à la ligne (alinea)

Bla\\
%saut de paragraphe

Bla

\newpage

\section{Problématique soulevée}

Bla

\begin{center}
Problématique du sujet
\end{center}

\section{Hypothèse de solution}

%Quoi :
Bla\\

Voici une liste :
\begin{itemize}
\item item 1;
\item item 2;
\item item 3;
\item item 4.
\end{itemize}

Bla\\

%Comment :
Bla

Bla\footnotemark\\

%Detail :
Bla(cf. ref. \cite{cite6}).
%citation référencé dans le document "bibliographie.bib" inclus à la fin du document

\footnotetext{Note bas de page "bla"}
