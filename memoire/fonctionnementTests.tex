\chapter{Fonctionnement et Tests}

Dans cette partie nous allons tout d'abord nous pencher sur le fonctionnement et les tests concernant la partie parser du projet, puis nous allons aborder les mêmes points pour la partie analyse des données.

\section{Parser}

Cette partie du projet est écrite en java, nous avons utilisé JUnit afin de mener à bien ces tests unitaires. De plus le dépôt git est muni d'un outil d'intégration continue (Travis) qui est assoscié à nos tests afin de garantir que le développement ne produit pas d'erreurs sur le fonctionnement du programme. \\

De plus nous avons effectué des tests de couverture en utilisant le plugin Cobertura Maven qui nous a aidé à détecter les points plus faibles au niveau des tests et ainsi proposer d'autres tests unitaires pour avoir une meilleur couverture.


\subsection{Politique de tests}

Du fait de l'utilisation de java, pour chaque classe non-utilitaire un premier test vérifie systématiquement que la création des objets fonctionne correctement, une fois ce critère validé, des tests spécifiques à chaque classe sont effectués en fonction des fonctionnalités qu'elles remplissent.\\

Le but de nos test est de verifier l'extraction des donnéees des logs, pour cela, nous avons choisi deux logs réels représentatifs de chaque version et avons appliqué la stratégie suivante:  \\
En premier lieu nous avons réalisé des test pour verifier la bonne detection de la version pour les différents types de version. \\
Ensuite nous avons pris chaque donnée de la structure et avons validé qu'elle ne soit pas vide, dans le cas où l'élément est une liste, nous vérifions que le nombre d'éléments correspond, et finalement pour les éléments carte/quantité, comme par exemple \textit{victory cards} ou \textit{market} nous vérifions que le couple nom, valeur corresponde.

\subsection{Tests unitaires}



\subsection{Tests de couverture}
%TODO: ajouter les images pour les tests de couverture
Nous n'avons pas pu effectuer de tests pour la classe LogHandler par manque de temps.\\ Le test de conversion du document dans le serveur MongoDb vers l'objet n'est pas nécesssaire car le parser se contente d'ajouter des éléments vers la base de données sans jamais récupérer des données déja écrite. Dans le cas ou l'évolution du projet amènerait à des lectures de données depuis la partie Java, il faudra créer ce test.

\section{Bugs rencontrés}

\paragraph*{Dictionnaire de Cartes} 

Il existe une classe qui sert à verifier l'existence d'une carte et returne en même temps le nom générique de la carte donnée.\\
Il y a des problèmes avec les cartes dont le pluriel change radicalement du singulier. par exemple les cartes Colonies (Colony) et Duchies (Duchy). \\

Pour resoudre le bug il faut changer le dictionaire actuelmement implementé, pour ajouter la convertion de cartes en pluriel qui sont des cas atypiques.

Pour la classe logReaderAbs, il y a des actions qui ne sont pas testées car, dans les logs selectionnés pour les tests il n'existe pas toutes les actions possibles pour les Joueurs, une solution serait de modifier le .

\paragraph*{Nom de Joueurs} 

Dans les logs nous avons trouvé des noms de joueurs très dificiles à traiter car ils ont beaucoup de caractères spéciaux ainsi que des noms explicitement ecrits pour casser des programmes,
par exemple System.out.println("hola"), ou ;) ou même , des noms avec des mots reservés pour les langages de programmation.


\section{Module d'analyse des donnée}
Cette partie du projet écrite en Python est cruciale car c'est ce qui sera utilisé par l'utilisateur. Nous avons utilisé le module unittest pour effectuer nos tests unitaires ainsi que mongomock pour simuler l'utilisation d'un serveur MongoDB. Les tests de couverture ont été effectués à l'aide de pytest

\subsection{Politique de tests}
Nous avons testé le module Match et Tools afin de vérifier leur bon fonctionnement, les tests effectués sont des tests unitaires. Pour effectuer nos tests, nous avons produit un log prédéfini a comparer au résultat des tests effectués.

\subsection{Tests unitaires}
\paragraph{module Match}
Trois tests unitaires ont été effectués:
\begin{itemize}
\item un test de comparaison du nom des joueurs.
\item un test de comparaison du log a proprement parler, c'est à dire les actions des joueurs au cours de la partie
\item un test de comparaison des informations du header.
\end{itemize}

Ces trois tests passent sans poser de problèmes.

\paragraph{Module Tools}
Trois tests unitaires ont été effectués:
\begin{itemize}
\item un premier test permet de vérifier la bonne crétation de la pseudo base de données (utilisation de mongomock).
\item un second test vérifie la bonne création des informations gloabales pour chaque joueur présent dans le log test (2 joueurs).
\item le troisième test vérifie que la fonction de calcul d'elo s'applique correctement ainsi que la détection du \textit{greening}.
\end{itemize}

\subsection{Tests de couverture}
Le résultat des tests de couverture est le suivant:\\
\includegraphics[scale=0.35,keepaspectratio]{./coverage_python_code}
Nous ne couvrons pas totalement le module Tools, cela s'explique par le fait que nous n'avons pas fait de tests sur la reconaissance des stratégies.\\
Le module MongoInterface n'est pas entièrement couvert non plus, car nous ne testons pas la création d'index et de collections.


\paragraph{conclusions}
Par manque de temps, nous n'avons pas pu effectuer plus de tests mais une suite logique à ces quelques tests unitaires serait de produire des logs pouvant poser problème, par exemple avec des caractères spéciaux ou bien des mots clefs relatifs au langage python. Concernant les test unitaires des outils, d'autres tests pourraient être effectués pour la detection des stratégies ou bien pour des valeurs d'ELO ambigües.



\subsection{Bugs rencontrés}
Du fait du peu de tests, nous n'avons pas rencontré de bugs sur ces tests mais les pistes évoquées dans la partie tests de couverture ne manqueront pas de révéler des faiblesses potentielles dans le code produit.

\iffalse
\subsection*{Sous-partie 1 ('apparaitra pas dans l'index)} Bla



\paragraph*{Paragraphe 3} Bla

\newpage

\subsection*{Sous-partie 2}

Bla

%galerie d'image
\begin{figure}[htp]
  \centering
  \subfloat[Première image]{\label{fig:première}\includegraphics[scale=0.8]{resultats/gallerie}}
  ~ %espace entre deux images sur une même ligne
  \subfloat[Deuxième image]{\label{fig:deuxième}\includegraphics[scale=0.8]{resultats/gallerie}}
  ~
  \subfloat[Troisième image]{\label{fig:troisième}\includegraphics[scale=0.8]{resultats/gallerie}}
  ~\\ %saute une ligne dans la galerie d'image
  \subfloat[Quatrième image]{\label{fig:quatrième}\includegraphics[scale=0.8]{resultats/gallerie}}
  ~
  \subfloat[Cinquième image]{\label{fig:cinquième}\includegraphics[scale=0.8]{resultats/gallerie}}
  \caption{Différents screenshots quelque chose, en gallerie}
  \label{fig:gallerie1}
\end{figure}
\fi
